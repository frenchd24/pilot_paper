\documentclass[twocolumn,tighten]{aastex6}
%DIF LATEXDIFF DIFFERENCE FILE
%DIF DEL french_article.tex             Mon Oct 17 16:06:34 2016
%DIF ADD french_article_revision1.tex   Fri Jan  6 12:12:13 2017
%DIF 2-4c2-3
%DIF < %\documentclass{aastex6}
%DIF < %\usepackage{emulateapj-rtx4}
%DIF < %\usepackage{emulateapj}
%DIF -------
 %DIF > 
\newcommand{\kms}{$\rm [km\, s^{-1}]$} %DIF > 
%DIF -------

 \shortauthors{French $\&$ Wakker}
\usepackage{graphicx}
\usepackage{subfigure}
%DIF PREAMBLE EXTENSION ADDED BY LATEXDIFF
%DIF UNDERLINE PREAMBLE %DIF PREAMBLE
\RequirePackage[normalem]{ulem} %DIF PREAMBLE
\RequirePackage{color}\definecolor{RED}{rgb}{1,0,0}\definecolor{BLUE}{rgb}{0,0,1} %DIF PREAMBLE
\providecommand{\DIFadd}[1]{{\protect\color{blue}\uwave{#1}}} %DIF PREAMBLE
\providecommand{\DIFdel}[1]{{\protect\color{red}\sout{#1}}}                      %DIF PREAMBLE
%DIF SAFE PREAMBLE %DIF PREAMBLE
\providecommand{\DIFaddbegin}{} %DIF PREAMBLE
\providecommand{\DIFaddend}{} %DIF PREAMBLE
\providecommand{\DIFdelbegin}{} %DIF PREAMBLE
\providecommand{\DIFdelend}{} %DIF PREAMBLE
%DIF FLOATSAFE PREAMBLE %DIF PREAMBLE
\providecommand{\DIFaddFL}[1]{\DIFadd{#1}} %DIF PREAMBLE
\providecommand{\DIFdelFL}[1]{\DIFdel{#1}} %DIF PREAMBLE
\providecommand{\DIFaddbeginFL}{} %DIF PREAMBLE
\providecommand{\DIFaddendFL}{} %DIF PREAMBLE
\providecommand{\DIFdelbeginFL}{} %DIF PREAMBLE
\providecommand{\DIFdelendFL}{} %DIF PREAMBLE
%DIF END PREAMBLE EXTENSION ADDED BY LATEXDIFF

\begin{document}

\title{Probing Large Galaxy Halos at $z\sim0$ with Automated $\rm L \MakeLowercase{y} \alpha$-Absorption Matching}

\author{David M. French, Bart P. Wakker}

\affil{Department of Astronomy, University of Wisconsin, Madison, WI 53706, USA}

\begin{abstract}

We present initial results from an ongoing large-scale study of the circumgalactic medium \DIFaddbegin \DIFadd{(CGM) }\DIFaddend in the nearby Universe ($cz \leq 10,000$ $\rm km\, s^{-1}$), using archival Cosmic Origins Spectrograph (COS) spectra of background \DIFdelbegin \DIFdel{QSOs}\DIFdelend \DIFaddbegin \DIFadd{quasi-stellar objects (QSOs)}\DIFaddend . This initial sample contains 33 \DIFdelbegin \DIFdel{sight lines }\DIFdelend \DIFaddbegin \DIFadd{sightlines }\DIFaddend chosen for their proximity to large galaxies ($D\geq25$ kpc) and high signal-to-noise ratio (S/N $\geq$ 10), yielding 48 Ly$\alpha$ systems which we have paired with individual galaxies. We introduce a likelihood parameter to facilitate the matching of galaxies to absorption lines in a reproducible manner. We find the usual anti-correlation between Ly$\alpha$ equivalent width ($EW$) and impact parameter ($\rho$) when we normalize by galaxy virial radius ($R_{vir}$). Galaxies associated with a Ly$\alpha$ absorber are found to be more highly-inclined than \DIFdelbegin \DIFdel{the average distribution of }\DIFdelend galaxies in the survey volume at a $>99\%$ confidence level (equivalent to $\sim 3.6 \sigma$ for a normal distribution). \DIFdelbegin \DIFdel{Contrary to }\DIFdelend \DIFaddbegin \DIFadd{In contrast with }\DIFaddend suggestions in other recent papers \DIFdelbegin \DIFdel{, we do not see obvious correlations }\DIFdelend \DIFaddbegin \DIFadd{of a correlation }\DIFaddend with azimuth angle \DIFaddbegin \DIFadd{for Mg\,{\sc ii} absorption, we find no such correlation for Ly$\alpha$}\DIFaddend .

\end{abstract}


\keywords{galaxies:intergalactic medium, galaxies:evolution, galaxies:halos, quasars: absorption lines}


\section{INTRODUCTION}

It is well known that galaxies must continue to accrete gas throughout their lifetimes in order to sustain their observed levels of star formation (e.g., Erb 2008, Prochaska $\&$ Wolfe 2009, Putman et al. 2009a, 2009b, Bauermeister et al. 2010, Genzel et al. 2010). This additional gas must come from the diffuse intergalactic medium (IGM), where the majority of the baryons in the universe reside (Penton et al. 2002, 2004; Lehner et al. 2007; Danforth $\&$ Shull 2008; Shull et al. 2012). How exactly this IGM gas eventually falls into the halos and disks of galaxies is still highly uncertain, as observational constraints are hard to come by. Because of the diffuse nature of IGM gas\DIFaddbegin \DIFadd{, }\DIFaddend it is most readily and sensitively detected as absorption in the spectra of background \DIFdelbegin \DIFdel{active galactic nuclei (AGN}\DIFdelend \DIFaddbegin \DIFadd{quasi-stellar objects (QSOs}\DIFaddend ). The advent of the sensitive \DIFdelbegin \DIFdel{UV spectrographs STIS and COS on the Hubble Space Telescope (HST) }\DIFdelend \DIFaddbegin \DIFadd{ultraviolet (UV) }\textbf{\textit{\DIFadd{Cosmic Origins Spectrograph}} \DIFadd{(COS) on the Hubble Space Telescope (HST; Osterman et al. 2011, Green et al. 2012)}} \DIFaddend has provided a wealth of information \DIFdelbegin \DIFdel{on }\DIFdelend \DIFaddbegin \DIFadd{about }\DIFaddend the properties and distribution of both the ions of heavy elements as well as the Lyman series of neutral \DIFaddbegin \DIFadd{hydrogen (}\DIFaddend H\,{\sc i}\DIFaddbegin \DIFadd{) }\DIFaddend gas around galaxies. 

Individual concentrations of gas along a given sightline imprint absorption lines \DIFdelbegin \DIFdel{on }\DIFdelend \DIFaddbegin \DIFadd{onto }\DIFaddend the spectrum in the direction of the QSO. The metal lines trace the star formation history within the intervening gas, and \DIFdelbegin \DIFdel{neutral hydrogen lines (}\DIFdelend \DIFaddbegin \DIFadd{H\,{\sc i} lines (e.g., }\DIFaddend Ly$\alpha$) indicate both the location and velocities of outflowing gas, as well as the presence of fuel for future star formation. Numerous studies using these observations have shown that many Ly$\alpha$ absorbers trace individual galaxy halos (e.g., Lanzetta et al 1995, Tripp et al. 1998, Chen et al. 1998, 2001a, Wakker $\&$ Savage 2009, Steidel et al. 2010, Prochaska et al. 2011, Thom et al 2012, Tumlinson et al. 2011 $\&$ 2013, Stocke et al. 2013 $\&$ 2014, Liang $\&$ Chen 2014, Tejos et al. 2014\DIFaddbegin \DIFadd{, }\textbf{\DIFadd{Borthakur et al. 2015}}\DIFaddend ).

Some recent studies find that about half of Ly$\alpha$ absorbers lie within galaxy \DIFdelbegin \DIFdel{haloes}\DIFdelend \DIFaddbegin \DIFadd{halos}\DIFaddend , at impact parameters $\rho<350$ kpc (C\^{o}t\'{e} et al. 2005, Prochaska et al. 2006, Wakker $\&$ Savage 2009). In addition, Wakker $\&$ Savage (2009) find that an absorber lies within 400 kpc and 400 $\rm km\, s^{-1}$ for $90\%$ of galaxies brighter than $0.1L_{\**}$, and all galaxies have a Ly$\alpha$ absorber within 1.5 Mpc. Higher redshift studies, such as Rudie et al. (2012a) at $2<z<3$, find evidence for an elevated density of absorbers up to 2 Mpc from galaxies. Wakker $\&$ Savage (2009) also discovered a correlation between Ly$\alpha$ absorption linewidth and impact parameter $\rho$, observing that the broadest lines (FWHM $>$150 $\rm km\, s^{-1}$) are only seen within 350 kpc of a galaxy, while at $\rho>1$ Mpc, only lines with FWHM $<75$ $\rm km\, s^{-1}$ occur. This suggests that the temperature and/or turbulence of gas increases in the presence of galaxies\DIFdelbegin \DIFdel{.
}\DIFdelend \DIFaddbegin \DIFadd{, a hypothesis that has been further supported by the results of Wakker et al. (2015). 
}\DIFaddend 

\DIFdelbegin \DIFdel{In addition, studying }\DIFdelend \DIFaddbegin \DIFadd{Studying }\DIFaddend the enrichment of galaxy halos is necessary for constraining outflow models and informing stellar feedback prescriptions. Directly measuring the \DIFdelbegin \DIFdel{velocity field }\DIFdelend \DIFaddbegin \DIFadd{velocities }\DIFaddend and column densities of absorbers as a function of impact parameter and orientation around galaxies would provide the clearest evidence of inflow or outflow activity, but results are still uncertain. Kacprzak et al. (2011b) claim to find that Mg\,{\sc ii} equivalent widths correlate with galaxy inclination, but Mathes et al. (2014) find no such correlation for Ly$\alpha$ and O\,{\sc vi} absorbers. Furthermore, we should expect outflowing gas to be more highly enriched and trace the metallicity of the associated galaxy, with inflowing gas instead appearing only in H\,{\sc i}. Both Stocke et al. (2013) and Liang $\&$ Chen (2014) find an ``edge'' to heavy ion absorption at $\sim0.5R_{vir}$, but \DIFdelbegin \DIFdel{with }\DIFdelend \DIFaddbegin \DIFadd{find }\DIFaddend Ly$\alpha$ covering fractions of $\sim0.75-1$ continuing out to $R_{vir}$. However, Mathes et al. (2014) measure O\,{\sc vi} absorption out to $\sim3 R_{vir}$, and Savage et al. (2014) find that more than half of O\,{\sc vi} absorption occurs beyond 1 $R_{vir}$ from the nearest galaxy. \DIFaddbegin \textbf{\DIFadd{Additionally, Borthakur et al. (2015) find that Ly$\alpha$ absorption $EW$ correlates with galaxy H\,{\sc i} gas fraction, but only weakly with SFR, suggesting that accretion flow from the CGM is slow and continuous.}}
\DIFaddend 

Recent results from Kacprzak et al. (2011b $\&$ 2012a) suggest that absorbing systems have a preferred orientation with respect to the major and minor axes of the galaxies they are associated with. This could be evidence of inflows and outflows, or an effect of the global structure of galaxy halos, but the statistics are not yet good enough to provide consistent answers. A larger-scale study of inclination and azimuthal angles vs. absorber properties is needed in order to elucidate the distribution of absorbing systems around galaxies. This is most easily done for the largest galaxies in the nearby universe, where it is possible to obtain inclinations and unambiguous absorber associations. 

Previous studies have suffered from small sample sizes (e.g., Mathes et al. 2014 use 14 galaxies, Stocke et al. 2013 use 11, Werk et al. 2014 use 44), \DIFdelbegin \DIFdel{and }\DIFdelend incompleteness due to their higher mean redshifts (e.g., the Mathes et al. 2014 sample is $0.12 <z<0.67$\DIFdelbegin \DIFdel{, and Werk et al. 2014 are complete to $\sim L^{\**}$ at $z\sim0.2$). }\DIFdelend \DIFaddbegin \DIFadd{), }\textbf{\DIFadd{and limited impact parameter reach (e.g., Werk et al. 2014 probe CGM gas only within $\rho < 160$ kpc of galaxies)}}\DIFadd{. }\DIFaddend To address these shortcomings, we are conducting a large survey of the properties of intergalactic gas in the nearby universe, where we have good and relatively complete information on both faint and bright galaxies, in order to reveal how the IGM and galaxies affect each other. 
\DIFaddbegin 

\DIFaddend We are taking advantage of the over 500 archived QSO and Seyfert spectra taken by the \DIFdelbegin \DIFdel{Cosmic Origins Spectrograph (COS ) }\DIFdelend \DIFaddbegin \DIFadd{COS }\DIFaddend and Space Telescope Imaging Spectrograph (STIS) on \DIFdelbegin \DIFdel{the Hubble Space Telescope (HST)}\DIFdelend \DIFaddbegin \DIFadd{HST}\DIFaddend , combined with the wealth of information available for the $\sim100,000$ galaxies with $cz<10,000$ $\rm km\, s^{-1}$ found in the NASA Extragalactic Database (NED) to probe the environment of absorbing gas systems in the nearby universe. \DIFdelbegin \DIFdel{This approach allows for an unbiased understanding of the distribution of the gas around galaxies, which requires looking for both detections and non-detections of gas, both near as well as far away from galaxies.
}\DIFdelend \DIFaddbegin \textbf{\DIFadd{In this paper we introduce a new, unbiased method for associating absorption lines with nearby galaxies. This approach will allow for an objective understanding of the distribution of the gas around galaxies, which requires looking for both detections and non-detections of gas, both near and far from galaxies, with a robust and reproducible metric for matching galaxies with absorption.}} 
\DIFaddend 

\DIFdelbegin \DIFdel{This paper presents initial results from our pilot study of 33 sight lines, chosen for their proximity to large galaxies and high signal-to-noise spectra. This paper }\DIFdelend \DIFaddbegin \textbf{\DIFadd{This paper presents our likelihood-matching method with initial results from a pilot study of 33 sight lines, chosen for their proximity to large galaxies and high signal-to-noise spectra.}} \DIFadd{It }\DIFaddend is organized as follows: in Section 2 we present the data and analysis techniques, in Section 3 we present the results, \DIFdelbegin \DIFdel{and }\DIFdelend in Section 4 we discuss possible interpretations of our results\DIFdelbegin \DIFdel{.
}\DIFdelend \DIFaddbegin \DIFadd{, }\textbf{\DIFadd{and Section 5 presents a summary.}}
\DIFaddend 

\begin{table*}[ht]\footnotesize
\begin{center}
\begin{tabular}{l l l l l l l l l l}
 \hline \hline
  Target 					& R.A. 		& Dec. 		 & \textit{z}	  & Program 	  & Grating 	   & Obs ID 	    & Obs Date 	     & $T_{exp}*$     & S/N*  \\ 
  	    					& 	       		&	  		 & 		  	  & 		    	  & 		  	   & 		  	    & 		     	     & 	        [ks]         & [1238] \\ 
 \scriptsize (1)  				& \scriptsize (2) & \scriptsize (3) & \scriptsize (4) & \scriptsize (5) & \scriptsize (6) & \scriptsize  (7) & \scriptsize (8) & \scriptsize (9) & \scriptsize (10)  \\ \hline \hline
\\

1H0717+714		  		&  \DIFdelbeginFL \DIFdelFL{7.0  21.0   }\DIFdelendFL \DIFaddbeginFL \DIFaddFL{07 21 }\DIFaddendFL 53.3  &  \DIFdelbeginFL \DIFdelFL{71.0  20.0  36.0  }\DIFdelendFL \DIFaddbeginFL \DIFaddFL{+71 20  36	}\DIFaddendFL &    0.5003	& 12025	&   G130M	&   LBG812	& 11-12-27      	 	  &  6.0    &      37         \\
2dFGRS\_S393Z082  		&  \DIFdelbeginFL \DIFdelFL{2.0  45.0    0.8   }\DIFdelendFL \DIFaddbeginFL \DIFaddFL{02 45 00.8  }\DIFaddendFL &  \DIFdelbeginFL \DIFdelFL{-30.0    7.0  23.0  }\DIFdelendFL \DIFaddbeginFL \DIFaddFL{$-$30 07  23	}\DIFaddendFL &    0.3392	& 12988	&   G130M	&   LC1040	& 13-05-27,28 	 	  & 17.7   &      10         \\
				  		&		       &			&			&		&			&    LC1045	&				  &	       &		   \\
FBQSJ1431+2442     		&  \DIFdelbeginFL \DIFdelFL{14.0  31.0  }\DIFdelendFL \DIFaddbeginFL \DIFaddFL{14 31 }\DIFaddendFL 25.8  &  \DIFdelbeginFL \DIFdelFL{24.0  42.0  20.0  }\DIFdelendFL \DIFaddbeginFL \DIFaddFL{+24 42 20	}\DIFaddendFL &   0.4069		& 13342	&   G130M	&   LC8903	& 15-03-29		  & 16.5  &      17          \\
				 		&		       &			&			& 12603	&			&   LBS314	& 13-03-08		  &	      &		  	  \\
H1101-232   		 		&  \DIFdelbeginFL \DIFdelFL{11.0  3.0   }\DIFdelendFL \DIFaddbeginFL \DIFaddFL{11 03 }\DIFaddendFL 37.7  &  \DIFdelbeginFL \DIFdelFL{-23.0  29.0  31.0  }\DIFdelendFL \DIFaddbeginFL \DIFaddFL{$-$23 29 31	}\DIFaddendFL &   0.1860		& 12025	&   G130M	&   LBG804	& 11-07-05  		   & 13.3  &      16         \\
HE0241-3043  		 		&  \DIFdelbeginFL \DIFdelFL{2.0  43.0  }\DIFdelendFL \DIFaddbeginFL \DIFaddFL{02 43 }\DIFaddendFL 37.7  &  \DIFdelbeginFL \DIFdelFL{-30.0  30.0  48.0  }\DIFdelendFL \DIFaddbeginFL \DIFaddFL{$-$30 30 48	}\DIFaddendFL &   0.6693		& 12988	&   G130M	&   LC1070	& 13-06-21  		   & 7.0    &      14         \\
LBQS1230-0015  	 		&  \DIFdelbeginFL \DIFdelFL{12.0  33.0  4.1  }\DIFdelendFL \DIFaddbeginFL \DIFaddFL{12 33 04.1  }\DIFaddendFL &  \DIFdelbeginFL \DIFdelFL{-0.0  31.0  34.0    }\DIFdelendFL \DIFaddbeginFL \DIFaddFL{$-$00 31 34    	}\DIFaddendFL &   0.4709		& 11598	&   G130M	&   LB5N15	& 10-08-01  		   & 10.3  &      13         \\
				 		&		       &			&			& 12486	&			&   LBP250	& 12-04-26		   &	       &	  	   \\
MRC2251-178  	 		&  \DIFdelbeginFL \DIFdelFL{22.0  54.0  5.9  }\DIFdelendFL \DIFaddbeginFL \DIFaddFL{22 54 05.9  }\DIFaddendFL &  \DIFdelbeginFL \DIFdelFL{-17.0  34.0  55.0  }\DIFdelendFL \DIFaddbeginFL \DIFaddFL{$-$17 34 55	}\DIFaddendFL &   0.0661		& 12029	&   G130M	&   LBGB03	& 11-09-29                   &  5.5   &      42         \\
Mrk290  					&  \DIFdelbeginFL \DIFdelFL{15.0  35.0  }\DIFdelendFL \DIFaddbeginFL \DIFaddFL{15 35 }\DIFaddendFL 52.3  &  \DIFdelbeginFL \DIFdelFL{57.0  54.0  9.0    }\DIFdelendFL \DIFaddbeginFL \DIFaddFL{+57 54 09	}\DIFaddendFL &   0.0296		& 11524	&   G130M	&   LB4Q02	& 09-10-28  		   &   3.9  &      38         \\
Mrk876  					&  \DIFdelbeginFL \DIFdelFL{16.0  13.0  }\DIFdelendFL \DIFaddbeginFL \DIFaddFL{16 13 }\DIFaddendFL 57.2  &  \DIFdelbeginFL \DIFdelFL{65.0  43.0  10.0   }\DIFdelendFL \DIFaddbeginFL \DIFaddFL{+65 43 10	}\DIFaddendFL &   0.1290		& 11524	&   G130M	&   LB4Q03	& 10-04-08,09  		   & 12.6  &      65         \\
				 		&		       &			&			& 11686	&			&   LB4F05	&				   &	       &	  	   \\
Mrk1014  					&  \DIFdelbeginFL \DIFdelFL{1.0  59.0  }\DIFdelendFL \DIFaddbeginFL \DIFaddFL{01 59 }\DIFaddendFL 50.2  &  \DIFdelbeginFL \DIFdelFL{0.0  23.0  41.0     }\DIFdelendFL \DIFaddbeginFL \DIFaddFL{+00 23 41	}\DIFaddendFL &   0.1630		& 12569	&   G130M	&   LBP404	& 12-01-25  		   &  1.8   &      17         \\
PG0832+251  				&  \DIFdelbeginFL \DIFdelFL{8.0  35.0  }\DIFdelendFL \DIFaddbeginFL \DIFaddFL{08 35 }\DIFaddendFL 35.9  &  \DIFdelbeginFL \DIFdelFL{24.0  59.0  41.0   }\DIFdelendFL \DIFaddbeginFL \DIFaddFL{+24 59 41	}\DIFaddendFL &   0.3310		& 12025	&   G130M	&   LBG808	& 12-04-19		   &  6.1   &      14         \\
PG0003+158  				&  \DIFdelbeginFL \DIFdelFL{0.0  5.0  }\DIFdelendFL \DIFaddbeginFL \DIFaddFL{00 05 }\DIFaddendFL 59.3  &  \DIFdelbeginFL \DIFdelFL{16.0  9.0  49.0    }\DIFdelendFL \DIFaddbeginFL \DIFaddFL{+16 09 49	}\DIFaddendFL &   0.4509		& 12038	&   G130M	&   LBGL17	& 11-10-22  		   & 10.4  &      25         \\
PG1001+054  				&  \DIFdelbeginFL \DIFdelFL{10.0  4.0  }\DIFdelendFL \DIFaddbeginFL \DIFaddFL{10 04 }\DIFaddendFL 20.1  &  \DIFdelbeginFL \DIFdelFL{5.0  13.0  1.0    }\DIFdelendFL \DIFaddbeginFL \DIFaddFL{+05 13 01	}\DIFaddendFL &   0.1610		& 13347	&   G130M	&   LCCV02	& 14-06-19  		   &  5.2   &      14         \\
						&		       &			&		 	& 13423	&			&   LC9W02	& 14-04-04		   &	       &	  	   \\
PG1302-102  				&  \DIFdelbeginFL \DIFdelFL{13.0  5.0  }\DIFdelendFL \DIFaddbeginFL \DIFaddFL{13 05 }\DIFaddendFL 33.0  &  \DIFdelbeginFL \DIFdelFL{-10.0  33.0  20.0  }\DIFdelendFL \DIFaddbeginFL \DIFaddFL{$-$10 33 20	}\DIFaddendFL &   0.2784  	& 12038	&   G130M	&   LBGL04	& 11-08-16  		   &  6.0   &      27         \\
RBS1768  				&  \DIFdelbeginFL \DIFdelFL{21.0  38.0  }\DIFdelendFL \DIFaddbeginFL \DIFaddFL{21 38 }\DIFaddendFL 49.7  &  \DIFdelbeginFL \DIFdelFL{-38.0  28.0  40.0  }\DIFdelendFL \DIFaddbeginFL \DIFaddFL{$-$38 28 40	}\DIFaddendFL &   0.1830  	& 12936	&   G130M	&   LC1201	& 13-06-25		   &  7.0   &      24         \\
RX J0714.5+7408  			&  \DIFdelbeginFL \DIFdelFL{7.0  14.0  }\DIFdelendFL \DIFaddbeginFL \DIFaddFL{07 14 }\DIFaddendFL 36.2  &  \DIFdelbeginFL \DIFdelFL{74.0  8.0  11.0   }\DIFdelendFL \DIFaddbeginFL \DIFaddFL{+74 08 11	}\DIFaddendFL &   0.3710  	& 12275	&   G130M	&   LBH402	& 11-03-18   		   &  8.3   &      18         \\
RX J1017.5+4702  			&  \DIFdelbeginFL \DIFdelFL{10.0  17.0  }\DIFdelendFL \DIFaddbeginFL \DIFaddFL{10 17 }\DIFaddendFL 30.9  &  \DIFdelbeginFL \DIFdelFL{47.0  2.0  25.0   }\DIFdelendFL \DIFaddbeginFL \DIFaddFL{+47 02 25	}\DIFaddendFL &   0.3354  	& 13314	&   G130M	&   LC9M04	& 14-01-29		   &  8.7   &      12         \\
RX J1117.6+5301  			&  \DIFdelbeginFL \DIFdelFL{11.0  17.0  }\DIFdelendFL \DIFaddbeginFL \DIFaddFL{11 17 }\DIFaddendFL 40.5  &  \DIFdelbeginFL \DIFdelFL{53.0  1.0  50.0   }\DIFdelendFL \DIFaddbeginFL \DIFaddFL{+53 01 50	}\DIFaddendFL &   0.1587  	& 14240	&   G130M	&   LCWM05	& 16-04-13		   &  4.9   &      11         \\
RX J1236.0+2641  			&  \DIFdelbeginFL \DIFdelFL{12.0  36.0  4.1  }\DIFdelendFL \DIFaddbeginFL \DIFaddFL{12 36 04.1  }\DIFaddendFL &  \DIFdelbeginFL \DIFdelFL{26.0  41.0  36.0   }\DIFdelendFL \DIFaddbeginFL \DIFaddFL{+26 41 36	}\DIFaddendFL &   0.2092  	& 12248	&   G130M	&   LBH087	& 12-01-29  		   &  4.2   &      11         \\
RX J1330.8+3119  			&  \DIFdelbeginFL \DIFdelFL{13.0  30.0  }\DIFdelendFL \DIFaddbeginFL \DIFaddFL{13 30 }\DIFaddendFL 53.2  &  \DIFdelbeginFL \DIFdelFL{31.0  19.0  32.0   }\DIFdelendFL \DIFaddbeginFL \DIFaddFL{+31 19 32	}\DIFaddendFL &   0.2423  	& 12248	&   G130M	&   LBHO85	& 11-07-11		   &   4.3  &      11         \\
RX J1356.4+2515  			&  \DIFdelbeginFL \DIFdelFL{13.0  56.0  }\DIFdelendFL \DIFaddbeginFL \DIFaddFL{13 56 }\DIFaddendFL 25.6  &  \DIFdelbeginFL \DIFdelFL{25.0  15.0  23.0   }\DIFdelendFL \DIFaddbeginFL \DIFaddFL{+25 15 23	}\DIFaddendFL &   0.1640  	& 12248	&   G130M	&   LBH057	& 12-02-03		   &   2.3  &      10         \\
RX J1503.2+6810  			&  \DIFdelbeginFL \DIFdelFL{15.0  3.0  }\DIFdelendFL \DIFaddbeginFL \DIFaddFL{15 03 }\DIFaddendFL 16.5  &  \DIFdelbeginFL \DIFdelFL{68.0  10.0  6.0     }\DIFdelendFL \DIFaddbeginFL \DIFaddFL{+68 10 06	}\DIFaddendFL &   0.1140  	& 12276	&   G130M	&   LBI609		& 10-12-31 		   &   1.9  &      11         \\
RX J1544.5+2827  			&  \DIFdelbeginFL \DIFdelFL{15.0  44.0  }\DIFdelendFL \DIFaddbeginFL \DIFaddFL{15 44 }\DIFaddendFL 30.5  &  \DIFdelbeginFL \DIFdelFL{28.0  27.0  56.0    }\DIFdelendFL \DIFaddbeginFL \DIFaddFL{+28 27 56	}\DIFaddendFL &   0.2314  	& 13423	&   G130M	&   LC9W08	& 14-02-25		   &   2.1  &      10         \\
RX J2043.1+0324  			&  \DIFdelbeginFL \DIFdelFL{20.0  43.0  6.2  }\DIFdelendFL \DIFaddbeginFL \DIFaddFL{20 43 06.2  }\DIFaddendFL &  \DIFdelbeginFL \DIFdelFL{3.0  24.0  50.0    }\DIFdelendFL \DIFaddbeginFL \DIFaddFL{+03 24 50	}\DIFaddendFL &   0.2710  	& 13840	&   G130M	&   LCJW02	& 14-10-23		   &   7.8  &      15         \\
RX J2139.7+0246  			&  \DIFdelbeginFL \DIFdelFL{21.0  39.0  }\DIFdelendFL \DIFaddbeginFL \DIFaddFL{21 39 }\DIFaddendFL 44.2  &  \DIFdelbeginFL \DIFdelFL{2.0  46.0  5.0     }\DIFdelendFL \DIFaddbeginFL \DIFaddFL{+02 46 05	}\DIFaddendFL &   0.2600  	& 13840	&   G130M	&   LCJW03	& 14-10-27  		   & 	7.9  &      16         \\
SBS0957+599         			&  \DIFdelbeginFL \DIFdelFL{10.0  1.0  2.6  }\DIFdelendFL \DIFaddbeginFL \DIFaddFL{10 01 02.6  }\DIFaddendFL &  \DIFdelbeginFL \DIFdelFL{59.0  44.0  15.0  }\DIFdelendFL \DIFaddbeginFL \DIFaddFL{+59 44 15	}\DIFaddendFL &   0.7475  	& 12248	&   G130M	&   LBHO65	& 11-03-18,19  		   &   3.3  &      12         \\
SDSSJ021218.32-073719.8  	&  \DIFdelbeginFL \DIFdelFL{2.0  12.0  }\DIFdelendFL \DIFaddbeginFL \DIFaddFL{02 12 }\DIFaddendFL 18.3  &  \DIFdelbeginFL \DIFdelFL{-7.0  37.0  20.0  }\DIFdelendFL \DIFaddbeginFL \DIFaddFL{$-$07 37 20	}\DIFaddendFL &   0.1739  	& 12248	&   G130M	&   LBHO83	& 11-06-26		   &   6.5  &      12         \\
				        	     	& 	  	       &			&    	  	 	&		&			&   LBHO92	& 11-08-21		   &          &                   \\
				        	     	& 	               &			&    	  	 	&		&			&   LBHO92	& 11-08-21		   &          &                   \\
SDSSJ080838.80+051440.0 	&  \DIFdelbeginFL \DIFdelFL{8.0  8.0  }\DIFdelendFL \DIFaddbeginFL \DIFaddFL{08 08 }\DIFaddendFL 38.8  &  \DIFdelbeginFL \DIFdelFL{5.0  14.0  40.0  }\DIFdelendFL \DIFaddbeginFL \DIFaddFL{+05 14 40	}\DIFaddendFL &   0.3606  	& 12603	&   G130M	&   LBS330	& 12-03-17  		   &   4.7  &      10         \\
SDSSJ091728.60+271951.0 	&  \DIFdelbeginFL \DIFdelFL{9.0  17.0  }\DIFdelendFL \DIFaddbeginFL \DIFaddFL{09 17 }\DIFaddendFL 28.6  &  \DIFdelbeginFL \DIFdelFL{27.0  19.0  51.0 }\DIFdelendFL \DIFaddbeginFL \DIFaddFL{+27 19 51	}\DIFaddendFL &   0.0756  	& 14071	&   G130M	&   LCX202	& 15-11-30		   & 15.5  &      11         \\
				        	     	& 	  	      &				&    	  	 	&		&			&   LCX2Z2	& 16-02-06		   &          &                   \\
SDSSJ112224.10+031802.0 	&  \DIFdelbeginFL \DIFdelFL{11.0  22.0  }\DIFdelendFL \DIFaddbeginFL \DIFaddFL{11 22 }\DIFaddendFL 24.1  &  \DIFdelbeginFL \DIFdelFL{3.0  18.0  2.0    }\DIFdelendFL \DIFaddbeginFL \DIFaddFL{+03 18 02	}\DIFaddendFL &   0.4753  	& 12603	&   G130M	&   LBS318	& 13-03-29		   &   7.6  &      13         \\
SDSSJ130524.30+035731.0 	&  \DIFdelbeginFL \DIFdelFL{13.0  5.0  }\DIFdelendFL \DIFaddbeginFL \DIFaddFL{13 05 }\DIFaddendFL 24.3  &  \DIFdelbeginFL \DIFdelFL{3.0  57.0  31.0   }\DIFdelendFL \DIFaddbeginFL \DIFaddFL{+03 57 31	}\DIFaddendFL &   0.5457  	& 12603	&   G130M	&   LBS321	& 12-06-25,26		   &   7.6  &      13         \\
SDSSJ135726.27+043541.4 	&  \DIFdelbeginFL \DIFdelFL{13.0  57.0  }\DIFdelendFL \DIFaddbeginFL \DIFaddFL{13 57 }\DIFaddendFL 26.2  &  \DIFdelbeginFL \DIFdelFL{4.0  35.0  41.0  }\DIFdelendFL \DIFaddbeginFL \DIFaddFL{+04 35 41	}\DIFaddendFL &   1.2345  	& 12264	&   G130M	&   LBJ005	& 11-06-22		   & 14.1  &      21         \\
				        	     	& 	  	      &				&    	  	 	&		&			&   LBJ007	& 11-06-26		   &          &                   \\
SDSSJ140428.30+335342.0 	&  \DIFdelbeginFL \DIFdelFL{14.0  4.0  }\DIFdelendFL \DIFaddbeginFL \DIFaddFL{14 04 }\DIFaddendFL 28.3  &  \DIFdelbeginFL \DIFdelFL{33.0  53.0  42.0  }\DIFdelendFL \DIFaddbeginFL \DIFaddFL{+33 53 42	}\DIFaddendFL &   0.5500  	& 12603	&   G130M	&  LBS320	& 13-03-03   		   &   7.7  &      10          \\
TON1009  			    	&  \DIFdelbeginFL \DIFdelFL{9.0  9.0  6.1   }\DIFdelendFL \DIFaddbeginFL \DIFaddFL{09 09 06.1  }\DIFaddendFL &  \DIFdelbeginFL \DIFdelFL{32.0  36.0  31.0  }\DIFdelendFL \DIFaddbeginFL \DIFaddFL{+32 36 31	}\DIFaddendFL &   0.8103  	& 12603	&   G130M	&  LBS328	& 12-04-22   		   &   4.7  &      12         \\

 \\
\hline

\end{tabular}
\end{center}
  \caption{\small{COS targets in this sample. *Total exposure time and S/N ratio is given for multi-orbit exposures.}}
  \label{target_table}
\end{table*}


\section{DATA AND ANALYSIS}

\subsection{Galaxy Data}
Achieving the goal of this study relies on knowing the locations and properties of all galaxies near detected Ly$\alpha$ absorption lines. To facilitate this, we have constructed a database of all $z\leq 0.033$ ($cz\leq 10,000$ $\rm km\, s^{-1}$) galaxies with published data available through the NASA Extragalactic Database (NED). A full description of this catalog will be presented in French $\&$ Wakker 2017 (in prep). Here we summarize its important aspects. 

\DIFdelbegin %DIFDELCMD < \begin{figure}[ht!]
%DIFDELCMD <         %%%
\DIFdelendFL \DIFaddbeginFL \begin{figure}[b!]
        \DIFaddendFL \centering
        \vspace{0pt}
        \includegraphics[width=0.50\textwidth]{fig1.pdf}
        \caption{\DIFdelbeginFL %DIFDELCMD < \small{Distribution of $L/L_{\**}$ values for all galaxies in the dataset. Black vertical lines highlight 1, 0.5, 0.1, 0.05 and 0.01 $L_{\**}$. The turnoff around 0.1$L_{\**}$ shows that on average, the dataset is mostly complete to 0.2$L_{\**}$.}%%%
\DIFdelendFL \DIFaddbeginFL \small{Distribution of $L/L_{\**}$ values for all galaxies in the dataset. Red dashed vertical lines highlight 1, 0.5, 0.1, 0.05 and 0.01 $L_{\**}$.}\DIFaddendFL }
        \label{completeness}
\end{figure} 

The galaxy dataset contains over 108,000 entries, and includes data from SDSS, 2MASS, 2dF, 6dF, RC3, and many other, smaller surveys. Our criteria for including a galaxy in this dataset is only an accurate, spectroscopic redshift which places the galaxy in the $cz \leq 10,000$ $\rm km\, s^{-1}$ velocity range. This restriction leads to a completeness limit of $B \lesssim 18.7$ mag, or $\sim0.2 L_*$, at $cz = 10,000$ $\rm km\, s^{-1}$, and progressively better towards lower velocities (see Figure \ref{completeness}). This limit will vary depending on which major surveys include a particular region of the sky. The major contributor is whether or not SDSS data is available, which begins around $cz = 5,000$ $\rm km\, s^{-1}$. Figure \ref{completeness} is split into 4 velocity bins to illustrate this. Our data is complete down to $\sim0.1 L_*$ in the first bin, $0 \leq cz \leq 2,500$ $\rm km\, s^{-1}$. At slightly higher velocity, $2500 \leq cz \leq 6000$ $\rm km\, s^{-1}$, the completeness falls to barely better than $\sim1.0 L_*$ as we move past the near and well studied galaxies, but have yet to reach the footprint of deep surveys. SDSS data becomes available in the last two bins, spanning $6000 \leq cz \leq 10,000$ $\rm km\, s^{-1}$, and correspondingly completeness remains high down to the SDSS limits of $B \lesssim 18.7$ mag, or $\sim0.2 L_*$ at $cz = 10,000$ $\rm km\, s^{-1}$.

Additionally, we have homogenized the galaxy data beyond the steps taken by NED by normalizing all measurements of galaxy inclination, position angle, and diameter to 2MASS $K$-band values. Most galaxies in NED have measures of inclination, position angle and diameter available in several different bands, so in order to make meaningful comparisons it is necessary to automatically choose one band for all measurements. We chose 2MASS values for this because it was an all-sky survey, and represents the largest fraction of available galaxy data. Physical galaxy diameters are derived from 2MASS \DIFdelbegin \DIFdel{$K_s$ }\DIFdelend \DIFaddbegin \DIFadd{$K_{s}$ }\DIFaddend ``total" angular diameter measurements and galaxy distances. 2MASS \DIFdelbegin \DIFdel{$K_s$ }\DIFdelend \DIFaddbegin \DIFadd{$K_{s}$ }\DIFaddend ``total" diameter estimates are surface brightness extrapolation measurements and are derived as 

\begin{equation}
r\DIFdelbegin \DIFdel{_{tot} }\DIFdelend \DIFaddbegin \DIFadd{_{\rm tot} }\DIFaddend = r' + a(\DIFdelbegin \DIFdel{ln(148)}\DIFdelend \DIFaddbegin \DIFadd{\ln(148)}\DIFaddend ^b,
\end{equation}

\noindent where $r_{tot}$ is defined as the point where the surface brightness extends to 5 disk scale lengths, $r'$ is the starting point radius ($>5" - 10"$ beyond the nucleus, or core influence), and $a$ and $b$ are Sersic exponential function scale length parameters ($f = f_0 \exp{(-r/a)}^{(1/b)}$, see Jarret et al. 2003 for a full description). Approximately $50\%$ of all the galaxies have this 2MASS \DIFdelbegin \DIFdel{$K_s$ }\DIFdelend \DIFaddbegin \DIFadd{$K_{s}$ }\DIFaddend ``total" diameter. Of the remainder, $20\%$ have SDSS diameters, $27\%$ have no published diameter, and $3\%$ have diameters from other surveys. We convert values in these other bands to 2MASS \DIFdelbegin \DIFdel{$K_s$ }\DIFdelend \DIFaddbegin \DIFadd{$K_{s}$ }\DIFaddend ``total" diameters via a simple least squares linear fit when necessary.

We used $B$-band magnitudes to estimate each galaxy's luminosity in units of $L_{\**}$ as follows:

\begin{equation}
	\frac{L}{L_{\**}} = 10\DIFdelbegin \DIFdel{^{-0.4 (M_{B} - M_{B_{\**}})}}\DIFdelend \DIFaddbegin \DIFadd{^{-0.4 (M_{\rm B} - M_{\rm B_{\**}})}}\DIFaddend .
\end{equation}

We adopt the CfA galaxy luminosity function by Marzke et al. (1994), which sets $B_{\**} $ = -19.57. Direct $B$ band measurements are available for $\sim 30\%$ of galaxies, and most of the rest have SDSS $g$ and $r$ magnitudes, which can be converted to $B$ via $B = g + 0.39 (g-r) + 0.21$ (Jester et al. 2005). Finally, we also compute an estimate of the virial radius of each galaxy as \DIFdelbegin \DIFdel{$log R_{vir} = 0.69 log D + 1.24$}\DIFdelend \DIFaddbegin \DIFadd{$\log R_{vir} = 0.69 \log D + 1.24$}\DIFaddend . This follows the parametrization of Stocke et al. (2013) relating a galaxy's luminosity to its virial radius, and the Wakker $\&$ Savage (2009) empirical relation between diameter and luminosity (see Wakker et al. 2015 and references therein for further details). Errors are propagated from the original published magnitude errors.

This homogeneous galaxy data table allows us to draw direct comparisons between the properties of the absorbers and the properties, separations, and environments of nearby galaxies with unprecedented completeness. The full dataset will be publicly released and discussed in further detail in a forthcoming paper (French \DIFdelbegin \DIFdel{et al. }\DIFdelend \DIFaddbegin \DIFadd{$\&$ Wakker }\DIFaddend 2017, in prep).


\subsection{Spectra}

This initial pilot study contains 33 sightlines to bright QSOs observed with COS. We chose sightlines by first sorting the galaxy data table described above by galaxy diameter. This sorted list is then correlated with the full list of publicly available sightlines, and only those systems with impact parameter less than 500 Mpc and galaxy diameter, $D$, greater than $25$ kpc  are kept. Finally, we select 33 sightlines with $S/N \geq 10$ from this list.

All COS spectra for the target sightlines were obtained through the Barbara A. Mikulski Archive for Space Telescopes (MAST), and processed with CALCOS v3.0 or later. We combined individual exposures by the method of Wakker et al. (2015), which corrects the COS wavelength scale by cross-correlating all ISM and IGM lines in each exposure. This method addresses the up to $\pm40$ $\rm km\, s^{-1}$ misalignments produced by CALCOS, and produces a corrected error array based on Poisson noise, which better matches the measured errors than the errors delivered in the x1d files. We then combine multiple exposures by aligning Galactic absorption lines with 21-cm spectra, and adding up the total counts in each pixel before converting to flux using the original, average flux-count ratio at each wavelength.


\begin{table*}[ht]\footnotesize
\begin{center}
\begin{tabular}{l l l l l l l l l l l l l l l}
 \hline \hline
 \DIFdelbeginFL \DIFdelFL{$Target$	}\DIFdelendFL \DIFaddbeginFL \DIFaddFL{$\rm Target$ }\DIFaddendFL & \DIFdelbeginFL \DIFdelFL{$Galaxy$  }\DIFdelendFL \DIFaddbeginFL \DIFaddFL{$\rm Galaxy$ }\DIFaddendFL & $R_{vir}$         & \DIFaddbeginFL \DIFaddFL{$\rm L/L_{\**}$ }& \DIFaddendFL $v_{galaxy}$    &  \DIFdelbeginFL \DIFdelFL{$Inc.$               }\DIFdelendFL \DIFaddbeginFL \DIFaddFL{$\rm Inc.$          }\DIFaddendFL &  \DIFdelbeginFL \DIFdelFL{$Az.$ 	       }\DIFdelendFL \DIFaddbeginFL \DIFaddFL{$\rm Az.$ 	     }\DIFaddendFL & $\rho$		& $v_{Ly\alpha}$	& $W_{Ly\alpha}$   & $\Delta v$  	     & $\mathcal{L}$  \\ 
  	   	     &                        & \scriptsize [kpc] &          	      \DIFaddbeginFL & \DIFaddendFL \scriptsize \DIFdelbeginFL \DIFdelFL{$\rm [km ~s^{-1}]$ }\DIFdelendFL \DIFaddbeginFL \kms \DIFaddendFL & \scriptsize [deg] & \scriptsize [deg] & \scriptsize [kpc] & \scriptsize \DIFdelbeginFL \DIFdelFL{$\rm [km\, s^{-1}]$ }\DIFdelendFL \DIFaddbeginFL \kms \DIFaddendFL & \scriptsize \DIFdelbeginFL \DIFdelFL{$\rm [km\, s^{-1}]$ }\DIFdelendFL \DIFaddbeginFL \kms \DIFaddendFL & \scriptsize \DIFdelbeginFL \DIFdelFL{$\rm [km\, s^{-1}]$ }\DIFdelendFL \DIFaddbeginFL \kms \DIFaddendFL & 		\\
\scriptsize (1) & \scriptsize (2) & \scriptsize (3) & \scriptsize (4) & \scriptsize (5)      & \scriptsize (6)     & \scriptsize  (7)   & \scriptsize (8)    & \scriptsize (9)        & \scriptsize (10)     & \scriptsize (11)    \DIFaddbeginFL & \scriptsize \DIFaddFL{(12) }\DIFaddendFL \\ \hline \hline

1H0717+714  				&  UGC03804  					&  173  & \DIFaddbeginFL \DIFaddFL{1.9 }&  \DIFaddendFL 2887  	&  55  &  7  	&  207  &  2870  	&  343$\pm$6  		&  17  	&  0.24   \\
1H0717+714  				&  UGC03804  					&  173  & \DIFaddbeginFL \DIFaddFL{1.9 }&  \DIFaddendFL 2887  	&  55  &  7  	&  207  &  2956  	&  39$\pm$4  		&  -69  	&  0.21  \\
2dFGRS\_S393Z082  		&  NGC1097  					&  304  & \DIFaddbeginFL \DIFaddFL{6.1 }&  \DIFaddendFL 1271  	&  58  &  27  	&  112  &  1239  	&  570$\pm$21  	&  32  	&  1.9*  \\
H1101-232  				&  MCG-04-26-019  				&  173  & \DIFaddbeginFL \DIFaddFL{1.1 }&  \DIFaddendFL 3623  	&  68  &  26  	&  179  &  3580  	&  573$\pm$12  	&  43  	&  0.33  \\
HE0241-3043  				&  NGC1097  					&  304  & \DIFaddbeginFL \DIFaddFL{6.1 }&  \DIFaddendFL 1271  	&  58  &  77  	&  219  &  1221  	&  83$\pm$12  		&  50  	&  1.6*  \\
HE0241-3043  				&  NGC1097  					&  304  & \DIFaddbeginFL \DIFaddFL{6.1 }&  \DIFaddendFL 1271  	&  58  &  77  	&  219  &  1310  	&  184$\pm$15  	&  -39  	&  1.6*  \\
LBQS1230-0015  			&  NGC4517  					&  208  & \DIFaddbeginFL \DIFaddFL{0.5 }&  \DIFaddendFL 1128  	&  90  &  90  	&  110  &  1127  	&  473$\pm$16  	&  1  		&  1.6*  \\
MRC2251-178  			&  MCG-03-58-009  				&  319  & \DIFaddbeginFL \DIFaddFL{2.3 }&  \DIFaddendFL 9030  	&  61  &  39  	&  320  &  9051  	&  60$\pm$4  		&  -21  	&  1.4*  \\
Mrk1014  					&  NGC0768  					&  231  & \DIFaddbeginFL \DIFaddFL{3.0 }&  \DIFaddendFL 7021  	&  64  &  85  	&  486  &  7080  	&  117$\pm$11  	&  -59  	&  0.042*  \\
Mrk290  					&  NGC5987  					&  322  & \DIFaddbeginFL \DIFaddFL{3.0 }&  \DIFaddendFL 3010  	&  67  &  12  	&  486  &  3105  	&  511$\pm$5  		&  -95  	&  0.77*  \\
Mrk290  					&  NGC5987  					&  322  & \DIFaddbeginFL \DIFaddFL{3.0 }&  \DIFaddendFL 3010  	&  67  &  12  	&  486  &  3207  	&  319$\pm$4  		&  -197  	&  0.37*  \\
Mrk876  					&  UGC10294  					&  165  & \DIFaddbeginFL \DIFaddFL{0.1 }&  \DIFaddendFL 3504  	&  51  &  7  	&  274  &  3478  	&  280$\pm$3  		&  26  	&  0.063  \\
PG0003+158  				&  NGC7814  					&  171  & \DIFaddbeginFL \DIFaddFL{1.2 }&  \DIFaddendFL 1050  	&  68  &  47  	&  197  &  833  		&  131$\pm$15  	&  217  	&  0.081  \\
PG0832+251  				&  KUG0833+252  				&  165  & \DIFaddbeginFL \DIFaddFL{0.7 }&  \DIFaddendFL 6964  	&  62  &  55  	&  294  &  6980  	&  133$\pm$14  	&  -16  	&  0.041  \\
PG0832+251  				&  KUG0833+252  				&  165  & \DIFaddbeginFL \DIFaddFL{0.7 }&  \DIFaddendFL 6964  	&  62  &  55  	&  294  &  7201  	&  48$\pm$10  		&  -237  	&  0.01  \\
PG1001+054  				&  UGC05432  					&  164  & \DIFaddbeginFL \DIFaddFL{1.3 }&  \DIFaddendFL 3995  	&  36  &  78  	&  217  &  4092  	&  222$\pm$10  	&  -97  	&  0.14  \\
PG1302-102  				&  NGC4939  					&  235  & \DIFaddbeginFL \DIFaddFL{4.4 }&  \DIFaddendFL 3110  	&  48  &  61  	&  265  &  3448  	&  71$\pm$5  		&  -338  	&  0.05*  \\
RBS1768  				&  RFGC3781  					&  253  & \DIFaddbeginFL \DIFaddFL{1.0 }&  \DIFaddendFL 9162  	&  90  &  74  	&  464  &  9360  	&  364$\pm$4  		&  -198  	&  0.056*  \\
RBS1768  				&  RFGC3781  					&  253  & \DIFaddbeginFL \DIFaddFL{1.0 }&  \DIFaddendFL 9162  	&  90  &  74  	&  464  &  9434  	&  160$\pm$5  		&  -272  	&  0.024*  \\
RX J0714.5+7408  			&  UGC03717  					&  202  & \DIFaddbeginFL \DIFaddFL{1.2 }&  \DIFaddendFL 4188  	&  63  &  83  	&  271  &  4074  	&  58$\pm$7  		&  114  	&  0.13*  \\
RX J0714.5+7408  			&  UGC03717  					&  202  & \DIFaddbeginFL \DIFaddFL{1.2 }&  \DIFaddendFL 4188  	&  63  &  83  	&  271  &  4264  	&  410$\pm$9  		&  -76  	&  0.15*  \\
RX J1017.5+4702  			&  NGC3198  					&  191  & \DIFaddbeginFL \DIFaddFL{1.3 }&  \DIFaddendFL 663  	&  73  &  55  	&  378  &  629  		&  60$\pm$17  		&  34  	&  0.02  \\
RX J1117.6+5301  			&  NGC3631 					&  187  & \DIFaddbeginFL \DIFaddFL{1.9 }&  \DIFaddendFL 1156  	&  16  &  47  	&  198  &  1131  	&  356$\pm$20  	&  25  	&  0.32  \\
RX J1117.6+5301  			&  NGC3631 	 				&  187  & \DIFaddbeginFL \DIFaddFL{1.9 }&  \DIFaddendFL 1156  	&  16  &  47  	&  198  &  1259  	&  57$\pm$17  		&  -103  	&  0.25  \\
RX J1236.0+2641  			&  NGC4559  					&  165  & \DIFaddbeginFL \DIFaddFL{0.7 }&  \DIFaddendFL 807  	&  64  &  31  	&  188  &  795  		&  295$\pm$37  	&  12  	&  0.27  \\
RX J1236.0+2641  			&  NGC4565  					&  292  & \DIFaddbeginFL \DIFaddFL{1.7 }&  \DIFaddendFL 1230  	&  90  &  39  	&  159  &  1012  	&  337$\pm$32  	&  218  	&  0.54*  \\
RX J1236.0+2641  			&  NGC4565  					&  292  & \DIFaddbeginFL \DIFaddFL{1.7 }&  \DIFaddendFL 1230  	&  90  &  39  	&  159  &  1188  	&  288$\pm$24  	&  42  	&  1.7*  \\
RX J1330.8+3119  			&  UGC08492  					&  204  & \DIFaddbeginFL \DIFaddFL{2.0 }&  \DIFaddendFL 7414  	&  16  &  41  	&  335  &  7401  	&  330$\pm$15  	&  13  	&  0.081*  \\
RX J1356.4+2515  			&  CGCG132-055  				&  206  & \DIFaddbeginFL \DIFaddFL{1.3 }&  \DIFaddendFL 8671  	&  36  &  25  	&  190  &  8475  	&  126$\pm$18  	&  196  	&  0.35*  \\
RX J1503.2+6810  			&  CGCG318-012  				&  250  & \DIFaddbeginFL \DIFaddFL{2.1 }&  \DIFaddendFL 9765  	&  52  &  1  	&  325  &  10122  	&  44$\pm$14  		&  -357  	&  0.031*  \\
RX J1544.5+2827  			&  CGCG166-047  				&  175  & \DIFaddbeginFL \DIFaddFL{1.8 }&  \DIFaddendFL 9646  	&  43  &  61  	&  326  &  9642  	&  183$\pm$14  	&  4  		&  0.031  \\
RX J1544.5+2827  			&  CGCG166-047  				&  175  & \DIFaddbeginFL \DIFaddFL{1.8 }&  \DIFaddendFL 9646  	&  43  &  61  	&  326  &  9759  	&  169$\pm$12  	&  -113  	&  0.023  \\
RX J2043.1+0324  			&  NGC6954  					&  166  & \DIFaddbeginFL \DIFaddFL{1.4 }&  \DIFaddendFL 4067  	&  56  &  66  	&  301  &  4080  	&  82$\pm$10  		&  -13  	&  0.037  \\
RX J2139.7+0246  			&  UGC11785  					&  203  & \DIFaddbeginFL \DIFaddFL{0.7 }&  \DIFaddendFL 4074  	&  90  &  69  	&  108  &  4083  	&  490$\pm$7  		&  -9  	&  1.5  \\
RX J2139.7+0246  			&  UGC11785  					&  203  & \DIFaddbeginFL \DIFaddFL{0.7 }&  \DIFaddendFL 4074  	&  90  &  69  	&  108  &  4181  	&  529$\pm$7  		&  -107  	&  1.2*  \\
SBS0957+599  			&  MCG+10-14-058  				&  261  & \DIFaddbeginFL \DIFaddFL{1.1 }&  \DIFaddendFL 9501  	&  75  &  19  	&  206  &  9469  	&  78$\pm$12  		&  32  	&  1.4*  \\
SDSSJ021218.32-073719.8  	&  SDSSJ021315.79-073942.7  	&  174  & \DIFaddbeginFL \DIFaddFL{1.8 }&  \DIFaddendFL 4800  	&  52  &  10  	&  268  &  4756  	&  528$\pm$15  	&  44  	&  0.09  \\
SDSSJ021218.32-073719.8  	&  SDSSJ021315.79-073942.7  	&  174  & \DIFaddbeginFL \DIFaddFL{1.8 }&  \DIFaddendFL 4800  	&  52  &  10  	&  268  &  4833  	&  500$\pm$17  	&  -33  	&  0.092  \\
SDSSJ080838.80+051440.0  	&  UGC04239  					&  279  & \DIFaddbeginFL \DIFaddFL{2.1 }&  \DIFaddendFL 8763  	&  45  &  38  	&  378  &  8740  	&  883$\pm$24  	&  23  	&  0.87*  \\
SDSSJ080838.80+051440.0  	&  UGC04239  					&  279  & \DIFaddbeginFL \DIFaddFL{2.1 }&  \DIFaddendFL 8763  	&  45  &  38  	&  378  &  8927  	&  130$\pm$19  	&  -164  	&  0.45*  \\
SDSSJ091728.60+271951.0  	&  UGC04895  					&  204  & \DIFaddbeginFL \DIFaddFL{2.1 }&  \DIFaddendFL 7073  	&  61  &  32  	&  408  &  7141  	&  374$\pm$23  	&  -68  	&  0.022*  \\
SDSSJ112224.10+031802.0  	&  NGC3640  					&  180  & \DIFaddbeginFL \DIFaddFL{2.8 }&  \DIFaddendFL 1251  	&  38  &  22  	&  139  &  1049  	&  288$\pm$30  	&  202  	&  0.4  \\
SDSSJ112224.10+031802.0  	&  NGC3640  					&  180  & \DIFaddbeginFL \DIFaddFL{2.8 }&  \DIFaddendFL 1251  	&  38  &  22  	&  139  &  1264  	&  424$\pm$27  	&  -13  	&  1.1  \\
SDSSJ130524.30+035731.0  	&  UGC08186  					&  268  & \DIFaddbeginFL \DIFaddFL{1.1 }&  \DIFaddendFL 7006  	&  82  &  14  	&  249  &  7039  	&  480$\pm$14  	&  -33  	&  1.3*  \\
SDSSJ135726.27+043541.4  	&  NGC5364  					&  211  & \DIFaddbeginFL \DIFaddFL{2.4 }&  \DIFaddendFL 1241  	&  57  &  84  	&  183  &  1124  	&  85$\pm$11  		&  117  	&  0.74*  \\
SDSSJ135726.27+043541.4  	&  NGC5364  					&  211  & \DIFaddbeginFL \DIFaddFL{2.4 }&  \DIFaddendFL 1241  	&  57  &  84  	&  183  &  1296  	&  98$\pm$9  		&  -55  	&  0.97*  \\
SDSSJ140428.30+335342.0  	&  KUG1402+341  				&  204  & \DIFaddbeginFL \DIFaddFL{1.1 }&  \DIFaddendFL 7919  	&  72  &  63  	&  118  &  7884  	&  889$\pm$28  	&  35  	&  1.4  \\
TON1009  				&  NGC2770  					&  204  & \DIFaddbeginFL \DIFaddFL{1.9 }&  \DIFaddendFL 1947  	&  87  &  41  	&  274  &  1961  	&  350$\pm$21 	&  -14  	&  0.19*  \\

 \\
\hline
\end{tabular}
\end{center}
  \caption{\DIFdelbeginFL %DIFDELCMD < \small{All associated systems. The largest $\mathcal{L}$ value is given, with a (\**) indicating that this corresponds to $\mathcal{L}_{d^{1.5}}$, otherwise the quoted $\mathcal{L}$ was computed with $R_{vir}$.}%%%
\DIFdelendFL \DIFaddbeginFL \small{All associated systems. The largest $\mathcal{L}$ value is given, with a (\**) indicating that this corresponds to $\mathcal{L}_{\rm D^{1.5}}$, otherwise the quoted $\mathcal{L}$ was computed with $R_{vir}$.}\DIFaddendFL }
  \label{target_table}
\end{table*}



\section{RESULTS}

We have identified 48 Ly$\alpha$ absorption lines in the spectra of our initial 33 QSO sample\DIFdelbegin \DIFdel{which can be unambiguously associated with a single nearby galaxy of diameter $D\geq25$ kpc. }\DIFdelend \DIFaddbegin \DIFadd{, }\textbf{\DIFadd{each of which has been associated with a single nearby galaxy of diameter $D\geq25$ kpc. Each absorption component is treated individually, resulting in several cases where multiple absorbers are associated with the same galaxy.}} \DIFaddend In order to be considered for a pairing, a galaxy and absorption feature must appear within 400 $\rm km\, s^{-1}$ in velocity and 500 kpc in physical impact parameter from each other. When multiple galaxies pass these criteria for a particular line, we are left with two options. 1) one galaxy is obviously far larger and closer in physical and velocity space to the \DIFdelbegin \DIFdel{line}\DIFdelend \DIFaddbegin \DIFadd{sightline}\DIFaddend , and may have several satellite galaxies, or 2) there are multiple galaxies near the absorber, making any association ambiguous; we do not include these cases in the further analysis.

\begin{figure}[t!]
\centering
  \DIFdelbeginFL %DIFDELCMD < \subfigure[]{\includegraphics[width=0.87\linewidth]{fig2.pdf}}{\label{line}}
%DIFDELCMD <   %%%
\DIFdelendFL \DIFaddbeginFL \subfigure[]{\includegraphics[width=0.87\linewidth]{fig2.pdf}\label{line}}
  \DIFaddendFL \subfigure[]{\includegraphics[width=1.\linewidth]{fig3.pdf}\label{impactmap}}
  \caption{\DIFdelbeginFL %DIFDELCMD < \small{a) An example of 2 Ly$\alpha$ lines found in the Mrk290 sightline at 3090 and 3192 . b) A map of \textit{all} galaxies within a 500 kpc impact parameter of target Mrk290 sightline and with velocity ($cz$) within 400 $\rm km\, s^{-1}$ of absorption detected at 3192 $\rm km\, s^{-1}$ (central black star). The galaxy NGC5987 ($v=3010$ $\rm km\, s^{-1}$, inclination = $65^{\circ}$) can be unambiguously paired with the Ly$\alpha$ absorption features at $v=3090, 3192$ $\rm km\, s^{-1}$ because it is the largest and closest galaxy in both physical and velocity space to the absorption feature.}%%%
\DIFdelendFL \DIFaddbeginFL \small{a) An example of 2 Ly$\alpha$ lines found in the Mrk290 sightline at 3090 and 3192 . b) A map of \textit{all} galaxies within a 500 kpc impact parameter of target Mrk290 sightline and with velocity ($cz$) within 400 $\rm km\, s^{-1}$ of absorption detected at 3192 $\rm km\, s^{-1}$ (central black star). The galaxy NGC5987 ($v=3010$ $\rm km\, s^{-1}$, inclination = $65^{\circ}$) has been paired with the Ly$\alpha$ absorption features at $v=3090, 3192$ $\rm km\, s^{-1}$ because it is the largest and closest galaxy in both physical and velocity space to the absorption feature.}\DIFaddendFL }
\vspace{5pt}
\end{figure}

\begin{figure*}[t]
\centering
\subfigure[]{\label{ew_vs_impact}\includegraphics[width=0.49\textwidth]{fig4.pdf}}
\subfigure[]{\label{ew_vs_impact_vir}\includegraphics[width=0.49\textwidth]{fig5.pdf}}
\caption{\DIFdelbeginFL %DIFDELCMD < \small{a) Equivalent width of each absorber as a function of impact parameter, $\rho$. b) Equivalent width as a function of $\rho/R_{vir}$. The anti-correlation is strongest when scaling $\rho$ by the galaxy virial radius. Absorbers are separated into red and blue-shifted samples based on $\Delta v$. Bins of mean $EW$ are overplotted in red-dashed, and blue-dotted lines for their respective samples.}%%%
\DIFdelendFL \DIFaddbeginFL \small{a) Equivalent width of each absorber as a function of impact parameter, $\rho$. b) Equivalent width as a function of $\rho/R_{vir}$. The anti-correlation is strongest when scaling $\rho$ by the galaxy virial radius. Absorbers are separated into red and blue-shifted samples based on $\Delta v$. Bins of mean $EW$ are overplotted in red-dotted, and blue-dashed lines for their respective samples.}\DIFaddendFL }
\vspace{5pt}
\end{figure*}


To facilitate this decision, we calculate the likelihood, $\mathcal{L}$, of every possible galaxy-absorber pairing as follows:

\begin{equation}
	\mathcal{L} = A e\DIFdelbegin \DIFdel{^{-(\frac{\rho}{R_{eff}})^2} }\DIFdelend \DIFaddbegin \DIFadd{^{-(\frac{\rho}{R_{\rm eff}})^2} }\DIFaddend e\DIFdelbegin \DIFdel{^{-(\frac{\Delta v}{200})^2}}\DIFdelend \DIFaddbegin \DIFadd{^{-(\frac{\Delta v}{v_{\rm norm}})^2}}\DIFaddend .
\end{equation}

\noindent Here $\rho$ is the physical impact parameter, $\Delta v$ the velocity difference between the absorber and the galaxy ($\Delta v = v_{galaxy} - v_{absorber}$), \DIFaddbegin \textbf{\DIFadd{$v_{norm}$ is the velocity normalization constant}}\DIFadd{, }\DIFaddend and $A$ is a factor included to increase the likelihood in the case that $\rho \leq R_{eff}$ (in which case $A = 2$, otherwise $A = 1$). \DIFaddbegin \textbf{\DIFadd{Many similar studies and simulations (e.g., Wakker $\&$ Savage 2009, Liang $\&$ Chen 2014, Mathes et al. 2014) suggest that Ly$\alpha$ absorbers lie within 400 $\rm km\, s^{-1}$ of their associated galaxies, so throughout this paper we adopt a halfway point of $v_{norm}$=200 $\rm km\, s^{-1}$. Future work will explore the result of varying this normalization parameter and making refinements such as, e.g., relating $v_{norm}$ to the galaxy's rotation velocity.}}
\DIFaddend 

We compute $\mathcal{L}$ for two different values of $R_{eff}$: $R_{vir}$, the virial radius of the galaxy, and \DIFdelbegin \DIFdel{$d^{1.5}$}\DIFdelend \DIFaddbegin \DIFadd{$D^{1.5}$}\DIFaddend , the major diameter of the galaxy to the power of 1.5. $\mathcal{L}$ computed with $R_{vir}$ is liable to select satellite galaxies instead of the larger hosts, so including a version with \DIFdelbegin \DIFdel{$d^{1.5}$ }\DIFdelend \DIFaddbegin \DIFadd{$D^{1.5}$ }\DIFaddend serves as a two-tiered selection system. An \DIFdelbegin \DIFdel{absorber-galaxy system }\DIFdelend \DIFaddbegin \DIFadd{absorber }\DIFaddend separated by 200 $\rm km\, s^{-1}$ in velocity and 1$R_{vir}$ \DIFdelbegin \DIFdel{would have $\mathcal{L} = 0.27$}\DIFdelend \DIFaddbegin \DIFadd{in impact parameter from a $D=30$ kpc galaxy would have $\mathcal{L}_{R_{vir}} = 0.27$ and $\mathcal{L}_{D^{1.5}} = 0.11$}\DIFaddend . In order for an absorber to be marked as ``associated" with a particular galaxy, we require that its $\mathcal{L}$ must be a factor of 5 larger than the next best possible association, and $\mathcal{L} \ge 0.01$ for at least one of $\mathcal{L}_{R_{vir}}$ or $\mathcal{L}_{D^{1.5}}$. We visually inspect systems with only one $\mathcal{L}$ meeting these criteria, and decide to reject or include it based on the complexity of the nearby galaxy environment. \DIFaddbegin \textbf{\DIFadd{In Table \ref{target_table} we quote only the largest value of $\mathcal{L}$, and use an asterisk to denote when this corresponds to $\mathcal{L}_{D^{1.5}}$.}}
\DIFaddend 

Figures \ref{line} and \ref{impactmap} show \DIFdelbegin \DIFdel{a clean }\DIFdelend \DIFaddbegin \DIFadd{an }\DIFaddend example of a Ly$\alpha$ absorption line with a map of its galaxy environment, showing an unambiguous pairing between the absorption features at $3090, 3192$ $\rm km\, s^{-1}$ toward Mrk290 and galaxy NGC5987 (\DIFdelbegin \DIFdel{$\mathcal{L} = 0.37$). Unless explicitly stated, all following analysis concerns similarly unambiguous ``associated" systems. 
}\DIFdelend \DIFaddbegin \DIFadd{$\mathcal{L}^* = 0.37$). }\textbf{\DIFadd{All analysis that follows concerns similarly ``associated" systems.}}
\DIFaddend 

Additionally, we split the absorber-galaxy catalog based on the velocity difference of the two, $\Delta v$. With this scheme, we refer to an absorber with a \textit{lower} velocity than the associated galaxy as \textit{blueshifted}, while an absorber with a \textit{higher} velocity is referred to as \textit{redshifted}. The rest of the results will be analyzed based upon this splitting. In all figures blue and red-shifted absorbers are represented as blue diamonds and red circles, respectively, and red diamonds correspond to systems where \textit{both} a red and blue-shifted absorber is detected. \DIFaddbegin \textbf{\DIFadd{We use open symbols for systems with $\rho \leq R_{vir}$.}}
\DIFaddend 


\begin{figure*}[ht]
\centering
\subfigure[]{\label{w_vir}\includegraphics[width=0.49\textwidth]{fig6.pdf}}
\subfigure[]{\label{impact_vir}\includegraphics[width=0.49\textwidth]{fig7.pdf}}
\caption{\DIFdelbeginFL %DIFDELCMD < \small{a) Equivalent width of each absorber as a function of the virial radius of the associated galaxy. The blue-dotted and red-dashed lines shows the average $EW$ in 50 kpc bins of impact parameter for the blueshifted and redshifted absorbers, respectively. b) Impact parameter to each absorber as a function of the virial radius of the associated galaxy. The blue-dotted and red-dashed lines shows the average impact parameter in 50 kpc bins of $R_{vir}$ for the blueshifted and redshifted absorbers, respectively. The black dashed line indicates the cutoff at $\rho/R_{vir} =2.14$ imposed by our $\mathcal{L}$ limit.}%%%
\DIFdelendFL \DIFaddbeginFL \small{a) Equivalent width of each absorber as a function of the virial radius of the associated galaxy. b) Impact parameter to each absorber as a function of the virial radius of the associated galaxy. The black dashed line indicates the cutoff at $\rho/R_{vir} =2.14$ imposed by our $\mathcal{L}$ limit.} \DIFaddFL{In each, the blue-dashed and red-dotted lines shows the average $EW$ in 50 kpc bins of impact parameter for the blueshifted and redshifted absorbers, respectively. }\DIFaddendFL }
\vspace{5pt}
\end{figure*}


\vspace{10pt}


\subsection{EW-$\rho$ Anti-correlation}
\DIFaddbegin \label{ew}

\DIFaddend Numerous previous studies have suggested that Ly$\alpha$ equivalent width ($EW$) is anti-correlated with impact parameter ($\rho$) to the nearest galaxy. We find a weak \DIFdelbegin \DIFdel{correlation}\DIFdelend \DIFaddbegin \DIFadd{anti-correlation}\DIFaddend , as shown in Figure \ref{ew_vs_impact}. However, \DIFaddbegin \textbf{\DIFadd{as Churchill et al. (2013a) also found with Mg \,{\sc ii} absorption}}\DIFadd{, }\DIFaddend we find a stronger anti-correlation when we normalize $\rho$ by $R_{vir}$. Figure \ref{ew_vs_impact_vir} shows this expected anti-correlation when plotting $EW$ vs $\rho/R_{vir}$. A possible explanation for this trend is that larger galaxies host larger, more physically extended CGM halos. We would thus expect the absorber $EW$ to also correlate positively with $R_{vir}$. Figure \ref{w_vir} shows $EW$ as a function of $R_{vir}$, with the blue-dashed and red-dotted lines show the average $EW$ in bins of 50 kpc of $R_{vir}$, showing little evidence of a correlation. However, by similary plotting $\rho$ as a function of $R_{vir}$, we instead find some evidence that absorbers around larger galaxies tend to be found at higher impact parameters. While we expect the upper-left quadrant of this figure to be sparsely populated (our likelihood-based method would tend not to choose small galaxies at large \DIFdelbegin \DIFdel{distances}\DIFdelend \DIFaddbegin \DIFadd{impact parameters}\DIFaddend ), it is unclear to us why the lower-right quadrant (large galaxies with absorbers at low impact parameter) is also sparsely populated. The full-sized sample at the completion of our study should provide a clearer picture.


\begin{figure}[h!]
        \centering
        \includegraphics[width=0.49\textwidth]{fig8.pdf}
        \caption{\DIFdelbeginFL %DIFDELCMD < \small{Equivalent width of each absorber as a function of the inclination angle of the associated galaxy. The dashed black line shows the mean $EW$ of all absorbers in bins of $15^{\circ}$.}%%%
\DIFdelendFL \DIFaddbeginFL \small{Equivalent width of each absorber as a function of the inclination angle of the associated galaxy. The black and dashed-grey lines show the mean and 90th percentile $EW$ of all absorbers in bins of $15^{\circ}$.}\DIFaddendFL }
        \label{ew_vs_inclination}
        \vspace{2pt}
\end{figure}

\begin{figure}[h!]
        \centering
        \includegraphics[width=0.49\textwidth]{fig9.pdf}
        \caption{\small{\textbf{Top: }Distribution of inclinations for all associated galaxies, split into red and blue shifted sets. \textbf{Bottom:} Distribution of inclinations of all observed galaxies in the $cz \leq 10,000$ $\rm km\, s^{-1}$ redshift range. The dashed line shows the inclination distribution for a truly random sample (i.e., no observational biases).}}
        \label{hist_inc}
        \vspace{2pt}
\end{figure}


\begin{figure}[ht!]
        \centering
        \includegraphics[width=0.49\textwidth]{fig10.pdf}
        \caption{\DIFdelbeginFL %DIFDELCMD < \small{Equivalent width as a function of the velocity separation between the galaxy and absorption line.}%%%
\DIFdelendFL \DIFaddbeginFL \small{Equivalent width as a function of the velocity separation between the galaxy and absorption line. }\DIFaddendFL }
        \label{W_veldif}
        \vspace{5pt}
\end{figure} 


\subsection{Inclination}
\DIFaddbegin \label{inclination}

\DIFaddend In this section we examine the inclinations of the associated galaxies compared to the distributions of absorbers. We correct for the finite thickness of galaxies, which causes $b/a$ to deviate from cos $i$ at high inclinations, by computing galaxy inclinations with the following formula from Heidmann et al. (1972a):

\begin{equation}
	\cos(i) = \sqrt{\frac{q^2 - q_0^2}{1 - q_0^2}},
	\label{incEq}
\end{equation}

\noindent where $q = b/a$, the ratio of the minor to major axis, and $q_0$ is the intrinsic axis ratio, set to $q_0 = 0.2$ for all galaxies (e.g., Jones, Davies, and Trewhella 1996). 6 of the 48 total absorbers are associated with E or S0 type galaxies, but we have chosen to keep the value of $q_0 = 0.2$ uniform throughout. The calculated values of cos$(i)$ we use for these galaxies are thus conservative under-estimates of their true inclinations.

Figure \ref{ew_vs_inclination} shows red and blueshifted absorbers' $EW$ plotted against the inclinations of their associated galaxies. We note that there is a clear excess of absorbers near galaxies of high inclination, with $77\%$ of redshifted and $73\%$ of blueshifted absorbers being associated with galaxies of $i \geq$ 50 deg, and only 3 absorbers associated with a galaxy of $i<35$. The solid-black and dashed-grey lines show mean and 90th percentile histograms, respectively, in bins of 12 deg. There does not appear to be much evolution of $EW$ across galaxy inclination, although a slight increase of mean $EW$ is possibly present towards higher inclinations.

In total $75\%$ of absorbers are associated with \DIFdelbegin \DIFdel{high }\DIFdelend \DIFaddbegin \DIFadd{highly }\DIFaddend inclined galaxies ($i \geq$ 50 deg). Only $56\%$ of all galaxies in the survey volume are highly inclined, indicating a preference for detecting absorption around inclined galaxies. Figure \ref{hist_inc} shows the distribution of galaxy inclinations for both the red and blue-shifted associated galaxies and all galaxies within the survey volume. We tested the difference between the full distribution of inclination angles for all galaxies in our survey volume and the distribution for all associated galaxies (red + blue-shifted absorbers) using the \DIFdelbegin \DIFdel{Anderson-Darling (AD) and Kolmogorov-Smirnov (KS) statistical distribution tests, yielding p-values of $KS_{p} =0.014$, and $AD_{p} = 0.00037$. Assuming the AD test produces the more accurate p-value estimation, then at a $99.96\%$ confidence level the }\DIFdelend \DIFaddbegin \textbf{\DIFadd{Anderson-Darling (AD) statistical distribution test, yielding a p-value of $AD_{p} = 0.00037$. Thus at a $99.96\%$ confidence level ($\sim 3.6 \sigma$ for a normal distribution)}} \DIFadd{the }\DIFaddend inclinations of our associated galaxies are not sampled from the average distribution of observed inclinations. Hence, we take this to mean that the shape of the CGM of these galaxies is not perfectly spheroidal. 

It is worth noting here that the observed distribution of galaxy inclinations is \emph{not} flat, as one might naively expect. The dashed line in Figure \ref{hist_inc} shows the distribution of observable inclinations for a random, uniform sample (i.e., a uniform distribution of $q=b/a$ values between $0.2 - 1.0$). There could be a number of effects contributing to the difference between this expected distribution and the observed (shown in green). If our sample is magnitude limited and we assume galaxies are mostly optically thin but with an very-thin optically thick component (e.g., a dust lane), then mostly face-on and mostly edge-on galaxies would be underrepresented due to surface-brightness and dust obscuration effects (e.g., see Jones, Davies and Trewhella 1996). It is possible that a similar effect is also responsible for the over-abundance of Ly$\alpha$ detections around highly inclined galaxies. If we assume a disky or oblate spheroid halo shape and a covering fraction below unity for the CGM, the probability of encountering a cloud near an inclined galaxy would increase due to the increased path-length through the halo. We will produce a model to test this and other possible explanations in Paper II, when we have the much larger \DIFdelbegin \DIFdel{, finished }\DIFdelend dataset available.


\subsection{Velocity Difference \rm($\Delta v$\rm)}
\DIFaddbegin \label{veldiff}
\DIFaddend 

We find evidence for an anti-correlation between absorber $EW$ and the velocity difference between the galaxy and the associated absorption, $\Delta v$. The mean and maximal $EW$ of absorption increases with decreasing $\Delta v$ (see Figure \ref{W_veldif}). In total, 32/48 ($67\%$) of absorbers are found within $\pm100$ $\rm km\, s^{-1}$. This $\pm100$ $\rm km\, s^{-1}$ threshold also applies to absorber $EW$, with only 1 absorber of $EW \geq 400$ found with $\Delta v > 100$ $\rm km\, s^{-1}$. Blueshifted absorbers are on average closer to their associated galaxy, with $\overline{\Delta v}_{blue} = 68\pm16$ $\rm km\, s^{-1}$, compared to $\overline{\Delta v}_{red}=-108\pm20$ $\rm km\, s^{-1}$ for the redshifted sample, and correspondingly have higher average equivalent width, $\overline{EW}_{blue}=329\pm52$ $\rm m\AA$ compared to $\overline{EW}_{red}=245\pm34$ \DIFdelbegin \DIFdel{. }\DIFdelend $\rm m\AA$

Additionally, of the 48 associated absorbers, 29 are matched with the same galaxy as another absorber (for a total of 14 unique galaxies in this subset). All but one of these cases involve two absorbers in the same sightline yet separated in velocity around a galaxy. 23/29 of these are oriented such that the higher $EW$ absorber has the smaller $\Delta v$, and the 6 others are close in either velocity or $EW$. The one galaxy with 3 associated absorbers, NGC1097, shows this trend across two sightlines as well, with absorbers at $\Delta v = 32$ $\rm km\, s^{-1}$ and $EW = 570$ $\rm m\AA$ towards 2dFGRS\_S393Z082, and $\Delta v = -39$ $\rm km\, s^{-1}$ and $EW = 184$ $\rm m\AA$ and $\Delta v = 50$ $\rm km\, s^{-1}$ and $EW = 83$ $\rm m\AA$ towards HE0241-3043.

This result is opposite what we might expect \DIFaddbegin \textbf{\DIFadd{selection effects associated with}} \DIFaddend our likelihood method to produce. Because $\mathcal{L}$ is small for both high $\Delta v$ and high $\rho / R_{vir}$, there should be mostly low $\rho / R_{vir}$ systems at high $\Delta v$. Low $\rho / R_{vir}$ systems should also have higher $EW$ on average, as evidenced in the $EW-\rho$ anti-correlation discusses above. Figure \ref{W_veldif} shows the opposite however, with only low $EW$ systems at high $\Delta v$. It must therefore be the case that $EW$ tends to anti-correlate with \textit{both} $\Delta v$ \textit{and} $\rho / R_{vir}$.


\subsection{Azimuth}
\DIFaddbegin \label{azimuth}
\DIFaddend 

In this section we examine properties of absorbers as a function of their azimuthal angle with respect to their associated galaxy. Azimuth is defined as the angle between the major axis of a galaxy and the vector connecting the absorption feature and the midpoint of the galaxy plane. Figure \ref{azimuth_illustration} illustrates this. 

The mean azimuth angle for blueshifted absorbers is $43\pm5^{\circ}$, and $49\pm5^{\circ}$ for redshifted absorbers. Figure \ref{azimuth_dist} shows the distribution of azimuth angles for both red and blue-shifted absorbers. Unlike the findings of Kacprzak et al. (2011b, 2012a), who find a bimodal distribution of Mg\,{\sc ii} absorbers around galaxies, our distributions of Ly$\alpha$ absorbers are generally consistent with a flat, or random distribution. There is possibly a slight overabundance of redshifted absorbers around $0^{\circ}$ (minor-axis) and blueshifted absorbers between $20-50^{\circ}$ (just off major-axis), but we cannot assign this observation much significance yet given the small sample size. We additionally find no significant correlation between azimuth angle and $EW$ or $\Delta v$. See \DIFdelbegin \DIFdel{figure }\DIFdelend \DIFaddbegin \DIFadd{Figure }\DIFaddend \ref{azimuthMap} for a map of the locations of absorbers relative to their associated galaxies, split between red and blue-shifted absorbers and into three bins of inclination.

These contrasting results may indicate a genuine difference between the properties of Ly$\alpha$ and metal lines, since the distribution of the latter is thought to be influenced by outflows, which may be focused along the minor axis.

\begin{figure}[ht!]
        \centering
        \includegraphics[width=0.49\textwidth]{fig11.pdf}
        \caption{\DIFdelbeginFL %DIFDELCMD < \small{Azimuth is the angle, $\alpha$, between the major axis of the galaxy, $a$, and a vector extending from the AGN target to the galaxy center. Image of NGC891 credit: Composite Image Data - Subaru Telescope (NAOJ), Hubble Legacy Archive, Michael Joner, David Laney (West Mountain Observatory, BYU); Processing - Robert Gendler.}%%%
\DIFdelendFL \DIFaddbeginFL \small{Azimuth is the angle, $\alpha$, between the major axis of the galaxy, $a$, and a vector extending from the QSO target to the galaxy center. Image of NGC891 credit: Composite Image Data - Subaru Telescope (NAOJ), Hubble Legacy Archive, Michael Joner, David Laney (West Mountain Observatory, BYU); Processing - Robert Gendler.}\DIFaddendFL }
        \label{azimuth_illustration}
        \vspace{5pt}
\end{figure} 

\begin{figure}[ht!]
        \centering
        \includegraphics[width=0.49\textwidth]{fig12.pdf}
        \caption{\small{The distributions of azimuth angles for red and blue-shifted samples, with the combined sample plotted in black. Azimuth $= 0$ corresponds to absorption detected along the projected major axis of the galaxy, and Azimuth $= 90$ is along the minor axis.}}
        \label{azimuth_dist}
        \vspace{5pt}
\end{figure}


\begin{figure*}[ht!]
        \centering
        \includegraphics[width=0.98\textwidth]{fig13.pdf}
        \caption{\small{A map of where each absorber was detected with respect to the associated galaxies, separated into three bins of inclination (illustrated by the gray ellipse in the bottom left corner of each plot). \textbf{LEFT:} Absorbers associated with galaxies of inclination $0 \leq Inc \le 40$, \textbf{CENTER:} $40 \leq Inc \le 65$, and \textbf{RIGHT: } $65 \leq Inc$. Blueshifted and redshifted absorbers are separated into the top (diamonds) and bottom (circles) panels, and the marker size is scaled with $EW$. The sightlines containing multiple absorbers associated with the same galaxy can be identified by their darker color.}}
        \label{azimuthMap}
        \vspace{5pt}
\end{figure*} 



\DIFdelbegin \section{\DIFdel{SUMMARY}}
%DIFAUXCMD
\addtocounter{section}{-1}%DIFAUXCMD
\DIFdelend \DIFaddbegin \textbf{\section{\DIFadd{DISCUSSION}}}
\DIFaddend 


\DIFdelbegin %DIFDELCMD < \begin{table}[ht]\footnotesize
%DIFDELCMD < \begin{center}
%DIFDELCMD < \begin{tabular}{l l l}
%DIFDELCMD <  \hline \hline
%DIFDELCMD <  %%%
\DIFdelFL{Statistic                				}%DIFDELCMD < &  %%%
\DIFdelFL{Blueshifted Absorbers   }%DIFDELCMD < &     %%%
\DIFdelFL{Redshifted Absorbers     }\DIFdelendFL \DIFaddbeginFL \textbf{\DIFaddFL{In this paper we have chosen to separate absorption systems into their individual components, whereas many comparable CGM studies have instead chosen to group together absorption components within some velocity window. While making the assumption that absorbers within some velocity interval are physically linked certainly has merit, it is also possible that absorption components close in velocity may in fact be physically distinct (see, e.g., Churchill et al. 2015). Indeed, in Section \ref{veldiff} we identified several systems where the $EW$ of components of a possible Ly$\alpha$ system individually anti-correlate with $\Delta v$. We will thus explore both methods in Paper II, where the much larger sample size will allow for a meaningful comparison.}}\DIFaddendFL \\
\DIFdelbeginFL %DIFDELCMD < \hline \hline
%DIFDELCMD <  %%%
\DIFdelFL{Number 	          			 		}%DIFDELCMD < &     	%%%
\DIFdelFL{22				}%DIFDELCMD < &	%%%
\DIFdelFL{26			}\DIFdelendFL \DIFaddbeginFL 

\textbf{\DIFaddFL{Restricting ourselves to low redshift systems has several benefits and consequences. We are able to extend our search for associated galaxies to larger impact parameters than many related, higher-redshift CGM studies because of the availability of galaxy data. However, due to the observed anti-correlation between $EW$ and impact parameter, this results in a larger fraction of low-$EW$ and low-column density absorbers in our sample. Hence, we are likely tracing a region of the CGM not entirely analogous to that traced by, e.g., Kacprzak et al. (2011b), Mathes et al. (2014), and Borthakur et al. (2015).}}

\textbf{\DIFaddFL{Studies focusing on metal-lines (e.g., Mg\,{\sc ii} and O\,{\sc vi}) are generally associated with high-column density Lyman Limit Systems (LLS) which again tend to originate closer to their host galaxies. Most of the Ly$\alpha$ absorbers in this work are low-column density (generally log$~N$(H\,{\sc i}) $\leq$ 14), and originate near or beyond 1 $R_{vir}$. At these distances we may actually be probing the interface between CGM associated with individual galaxies and the larger-scale network of intergalactic gas filaments, thus the lack of any correlation with azimuth angle is not wholly unexpected.}}

\textbf{\DIFaddFL{We do, however, detect an inclination effect on the density, and possibly $EW$, of Ly$\alpha$ absorbers. The combination of no azimuthal dependence and increased absorber density with inclination leads us to conclude that these galaxies have disc-like, oblate-spheroidal halos. A perfectly spheroidal halo would show no correlation for either, and an extremely flattened halo would show up as enhanced number density along the major axis. These results are consistent with a picture where H\,{\sc i} covering fraction steadily decreases from $\sim$unity very near to galaxy discs out to at least 1 Mpc, where gas associated with galaxies merges with the general IGM. Our larger upcoming dataset will provide the statistics necessary to probe this in finer detail, as well as give clues to the exact shape of this falloff and the level of clumpiness or filamentary structure in galaxies H\,{\sc i} halos.}}\DIFaddendFL \\
\DIFdelbeginFL \DIFdelFL{Mean $EW$    }%DIFDELCMD < \scriptsize %%%
\DIFdelFL{$\rm [m\AA]$    }%DIFDELCMD < &	%%%
\DIFdelFL{$329 \pm 52$ 		}%DIFDELCMD < &	%%%
\DIFdelFL{$245 \pm 34$  	}%DIFDELCMD < \\
%DIFDELCMD <  %%%
\DIFdelFL{Median EW     }%DIFDELCMD < \scriptsize %%%
\DIFdelFL{$\rm [m\AA]$    }%DIFDELCMD < & 	%%%
\DIFdelFL{$292 \pm 16$		}%DIFDELCMD < & 	%%%
\DIFdelFL{$177 \pm 10$	}%DIFDELCMD < \\
%DIFDELCMD <  %%%
\DIFdelFL{Mean $\rm R_{vir}$   }%DIFDELCMD < \scriptsize [%%%
\DIFdelFL{kpc}%DIFDELCMD < ]	&   	%%%
\DIFdelFL{$215 \pm 10$		}%DIFDELCMD < & 	%%%
\DIFdelFL{$224 \pm 10$	}%DIFDELCMD < \\
%DIFDELCMD <  %%%
\DIFdelFL{Mean $\rho$   }%DIFDELCMD < \scriptsize [%%%
\DIFdelFL{kpc}%DIFDELCMD < ]          		&   	%%%
\DIFdelFL{$218 \pm 17$ 		}%DIFDELCMD < & 	%%%
\DIFdelFL{$298 \pm 23$	}%DIFDELCMD < \\
%DIFDELCMD <  %%%
\DIFdelFL{Mean $\Delta v$  }%DIFDELCMD < \scriptsize %%%
\DIFdelFL{$\rm [km\, s^{-1}]$     }%DIFDELCMD < &	%%%
\DIFdelFL{$ 68 \pm 16$    }%DIFDELCMD < &	%%%
\DIFdelFL{$-108 \pm 20$	}%DIFDELCMD < \\
%DIFDELCMD <  %%%
\DIFdelFL{Mean Inc.  }%DIFDELCMD < \scriptsize [%%%
\DIFdelFL{deg}%DIFDELCMD < ]  			&  	%%%
\DIFdelFL{$58 \pm 4$		}%DIFDELCMD < &	%%%
\DIFdelFL{$61 \pm 4$	}%DIFDELCMD < \\
%DIFDELCMD <  %%%
\DIFdelFL{Mean Az.  }%DIFDELCMD < \scriptsize [%%%
\DIFdelFL{deg}%DIFDELCMD < ]    			&	%%%
\DIFdelFL{$43 \pm 5$ 		}%DIFDELCMD < &	%%%
\DIFdelFL{$49 \pm 5$ 	}%DIFDELCMD < \\
%DIFDELCMD <   %%%
\DIFdelendFL 


\DIFdelbeginFL %DIFDELCMD < \hline
%DIFDELCMD < \end{tabular}
%DIFDELCMD < \end{center}
%DIFDELCMD <   %%%
%DIFDELCMD < \caption{%
{%DIFAUXCMD
%DIFDELCMD < \small{Average properties of the associated galaxy sample split into red and blue-shifted bins based on $\Delta v$.}%%%
}
  %DIFAUXCMD
%DIFDELCMD < \label{resultsTable}
%DIFDELCMD < \end{table}
%DIFDELCMD < %%%
\DIFdelend \DIFaddbegin \textbf{\section{\DIFadd{SUMMARY}}}
\DIFaddend 

\DIFdelbegin \DIFdel{We have measured 48 $\rm Ly\alpha$ absorption lines in the spectra of 33 COS targets and matched each to a single, large ($D\geq 25$ kpc) galaxy. Table \ref{resultsTable} presents a breakdown of our results when separating absorber-galaxy pairs into red and blue-shifted samples. The following summarizes our findings:
}\DIFdelend \DIFaddbegin \textbf{\DIFadd{We have introduced a novel, unbiased likelihood-method for associating absorption systems with nearby galaxies, and explored its implementation with a small subsample of 33 COS sightlines. Associating CGM absorbers with individual galaxies remains a difficult and ambiguous affair, but with this new metric we can at least do so in a reproducible manner. }}
\DIFaddend 

\DIFaddbegin \textbf{\DIFadd{In this pilot sample we have measured 48 $\rm Ly\alpha$ absorption lines in the spectra of 33 COS targets and matched each to a single, large ($D\geq 25$ kpc) galaxy. Table \ref{resultsTable} presents a breakdown of our results when separating absorber-galaxy pairs into red and blue-shifted samples. The following summarizes our findings:}}

\DIFaddend \vspace{10pt}

\indent \textbullet \indent We introduce a likelihood parameter\DIFaddbegin \DIFadd{, $\mathcal{L}$, }\DIFaddend based on Gaussian profiles centered around \DIFdelbegin \DIFdel{$\rho / R_{vir}$ and $\Delta v$ }\DIFdelend \DIFaddbegin \DIFadd{$\rho / R_{eff}$ and $\Delta v / v_{norm}$ }\DIFaddend to automate the matching of absorbers with associated galaxies. \DIFaddbegin \textbf{\DIFadd{The response of $\mathcal{L}$ can be tailored by choosing different values for $R_{eff}$ and $v_{norm}$ (we used $R_{eff}$ = }[\DIFadd{$R_{vir}$, $D^{1.5}$}] \DIFadd{and $v_{norm}$ = 200 $\rm km\, s^{-1}$ in this work, and will explore other parameterizations in a future paper).}}
\DIFaddend 

\vspace{10pt}

\indent \textbullet \indent \DIFaddbegin \DIFadd{Equivalent width (}\DIFaddend $EW$\DIFaddbegin \DIFadd{) }\DIFaddend anti-correlates most strongly with $\rho$ when normalized by $R_{vir}$. It follows that $EW$ weakly correlates and anti-correlates with $R_{vir}$ and $\rho$, respectively.

\vspace{10pt}

\indent \textbullet \indent The mean and maximal $EW$ of absorbers increases with decreasing $\Delta v$. The strongest absorbers are nearly all found within $\Delta v = \pm 100$ $\rm km\, s^{-1}$ of their associated galaxies.

\vspace{10pt}

\indent \textbullet \indent We find a slight dichotomy in the $EW$ of absorption blue-ward vs red-ward of associated galaxies. Redshifted absorbers are weaker, with $\overline{EW}$ = $245 \pm 34$ $\textrm{m\AA}$ compared to $EW$ = $329 \pm 52$ $\textrm{m\AA}$ for blueshifted absorbers. \DIFaddbegin \textbf{\DIFadd{Our present dataset is too small, however, to assign this difference much significance.}}
\DIFaddend 

\vspace{10pt}

\textbullet \indent $\rm Ly\alpha$ absorbers are most \DIFaddbegin \DIFadd{commonly }\DIFaddend associated with inclined galaxies. $73\%$ of blueshifted and $77\%$ of redshifted absorbers are associated with galaxies with $i \geq 50$ deg, whereas $56\%$ of all galaxies in the survey volume have similarly high inclinations. The distributions of associated vs all galaxy inclinations differ at a greater than $99\%$ confidence, or \DIFdelbegin \DIFdel{$3\sigma$}\DIFdelend \DIFaddbegin \DIFadd{$\sim 3.6\sigma$}\DIFaddend , level according to the Anderson-Darling distribution test.

\vspace{10pt}

\indent \textbullet \indent We find no strong evidence for azimuth preference for absorption - Ly$\alpha$ absorbers appear to be distributed uniformly around \DIFdelbegin \DIFdel{galaxies}\DIFdelend \DIFaddbegin \DIFadd{galaxy major and minor axes}\DIFaddend .

\DIFaddbegin \vspace{10pt}

\textbf{\DIFadd{In a future paper we will apply this method to a sample of hundreds of COS sightlines in an effort to produce the most statistically robust CGM study to date.}}


\begin{table}[ht]\footnotesize
\begin{center}
\begin{tabular}{l l l}
 \hline \hline
 \DIFaddFL{Statistic                				}&  \DIFaddFL{Blueshifted Absorbers   }&     \DIFaddFL{Redshifted Absorbers     }\\ 
  \hline \hline
 \DIFaddFL{Number 	          			 		}&     	\DIFaddFL{22				}&	\DIFaddFL{26			}\\
 \DIFaddFL{Mean $EW$    }\scriptsize \DIFaddFL{$\rm [m\AA]$    }&	\DIFaddFL{$329 \pm 52$ 		}&	\DIFaddFL{$245 \pm 34$  	}\\
 \DIFaddFL{Median EW     }\scriptsize \DIFaddFL{$\rm [m\AA]$    }& 	\DIFaddFL{$292 \pm 16$		}& 	\DIFaddFL{$177 \pm 10$	}\\
 \DIFaddFL{Mean $\rm R_{vir}$   }\scriptsize [\DIFaddFL{kpc}]	&   	\DIFaddFL{$215 \pm 10$		}& 	\DIFaddFL{$224 \pm 10$	}\\
 \DIFaddFL{Mean $\rho$   }\scriptsize [\DIFaddFL{kpc}]          		&   	\DIFaddFL{$218 \pm 17$ 		}& 	\DIFaddFL{$298 \pm 23$	}\\
 \DIFaddFL{Mean $\Delta v$  }\scriptsize \DIFaddFL{$\rm [km\, s^{-1}]$     }&	\DIFaddFL{$ 68 \pm 16$    }&	\DIFaddFL{$-108 \pm 20$	}\\
 \DIFaddFL{Mean Inc.  }\scriptsize [\DIFaddFL{deg}]  			&  	\DIFaddFL{$58 \pm 4$		}&	\DIFaddFL{$61 \pm 4$	}\\
 \DIFaddFL{Mean Az.  }\scriptsize [\DIFaddFL{deg}]    			&	\DIFaddFL{$43 \pm 5$ 		}&	\DIFaddFL{$49 \pm 5$ 	}\\

\hline
\end{tabular}
\end{center}
  \caption{\small{Average properties of the associated galaxy sample split into red and blue-shifted bins based on $\Delta v$. \textbf{All reported errors are standard errors in the mean}}}
  \label{resultsTable}
\end{table}



\DIFaddend \acknowledgements

\DIFaddbegin \textbf{\DIFadd{We would like to thank the referee for his/her valuable comments and suggestions.}} \DIFaddend This research has made use of the NASA/IPAC Extragalactic Database (NED) which is operated by the Jet Propulsion Laboratory, California Institute of Technology, under contract with the National Aeronautics and Space Administration. Based on observations with the NASA/ESA \textit{Hubble Space Telescope}, obtained at the Space Telescope Science Institute (STScI), which is operated by the Association of Universities for Research in Astronomy, Inc., under NASA contract NAS 5-26555. Spectra were retrieved from the Barbara A. Mikulski Archive for Space Telescopes (MAST) at STScI. Over the course of this study, D.M.F. and B.P.W. were supported by grant AST-1108913, awarded by the US National Science Foundation, and by NASA grants \textit{HST}-AR-12842.01-A, \textit{HST}-AR-13893.01-A, and \textit{HST}-GO-14240 (STScI).

\facility{HST (COS)}


\nocite{*}
\DIFdelbegin %DIFDELCMD < \bibliography{paper_bib}
%DIFDELCMD < %%%
\DIFdelend \DIFaddbegin \bibliography{french_bib2}
\DIFaddend \bibliographystyle{aasjournal}

\end{document}
